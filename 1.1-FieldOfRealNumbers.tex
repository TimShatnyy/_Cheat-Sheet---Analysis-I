% \tsThe{1.1.1 Lindemann, 1882}{
%     There is no equation of the form
%     \[
%         x^n + a_{n-1}x^{n-1} + \cdots + a_0 = 0,
%         \qquad a_i \in \mathbb{Q},
%     \]
%     for which $x = \pi$ is a solution.
% }
% \\
% \tsIdea{R 1.1.3}{
%     In A3 the element $y$ is uniquely determined and is denoted by $-x$.
% }
% \\
\tsIdea{Axioms of Addition}{
    \textbf{Axioms:}
    \begin{align*}
        \textbf{A1 (Associativity)} \quad   & x + (y + z) = (x + y) + z &  & \forall x,y,z \in \mathbb{R}                           \\
        \textbf{A2 (Neutral element)} \quad & x + 0 = x                 &  & \forall x \in \mathbb{R}                               \\
        \textbf{A3 (Inverse element)} \quad & x + (-x) = 0              &  & \forall x \in \mathbb{R}\ \exists (-x)  \in \mathbb{R} \\
        \textbf{A4 (Commutativity)} \quad   & x + y = y + x             &  & \forall x,y \in \mathbb{R}
    \end{align*}
}
\tsIdea{Axioms of Multiplication}{
    \textbf{Axioms:}
    \begin{align*}
        \textbf{M1 (Associativity)} \quad   & x\cdot(y\cdot z) = (x\cdot y)\cdot z &  & \forall x,y,z \in \mathbb{R}                                            \\
        \textbf{M2 (Neutral element)} \quad & x\cdot 1 = x                         &  & \forall x \in \mathbb{R}                                                \\
        \textbf{M3 (Inverse element)} \quad & x\cdot x^{-1} = 1                    &  & \forall x \in \mathbb{R}\setminus \{0\} \ \exists x^{-1} \in \mathbb{R} \\
        \textbf{M4 (Commutativity)} \quad   & x\cdot z = z\cdot x                  &  & \forall x,z \in \mathbb{R}
    \end{align*}
}
% \tsIdea{R 1.1.4}{
%     In M3 the element $y$ is uniquely determined and is denoted by $x^{-1}$.
% }
\\
\tsIdea{Distributivity (Compatibility of addition and mult.)}{
    \textbf{Axiom:}
    \[
        \textbf{D}\qquad x\cdot(y+z) = xy + xz = (y+z)\cdot x \qquad \forall x,y,z \in \mathbb{R}
    \]
}
\tsIdea{Order Axioms}{
    \textbf{Axioms:}
    \begin{align*}
        \textbf{O1 (Reflexivity)} \quad  & x \le x                                                   &  & \forall x \in \mathbb{R} \\
        \textbf{O2 (Transitivity)} \quad & (x \le y \land y \le z) \Rightarrow x \le z                                             \\
        \textbf{O3 (Antisymmetry)} \quad & (x \le y \land y \le x) \Rightarrow x = y                                               \\
        \textbf{O4 (Totality)} \quad     & \forall x,y \in \mathbb{R}:\ x \le y \ \text{or}\ y \le x
    \end{align*}
}
\tsIdea{Compatibility (Consistency with addition and mult.)}{
    \textbf{Axioms:}
    \begin{align*}
        \textbf{K1} \quad & x \le y \Rightarrow x+z \le y+z        &  & \forall x,y,z \in \mathbb{R} \\
        \textbf{K2} \quad & x \ge 0,\ y \ge 0 \Rightarrow xy \ge 0 &  & \forall x,y \in \mathbb{R}
    \end{align*}
}
\tsIdea{R 1.1.5}{
    The set $\mathbb{Q}$ of rational numbers, equipped with addition, multiplication, and the order relation $\le$, satisfies the above axioms.
}
\tsIdea{V (Completeness axiom)}{
    The set \(\mathbb{R}\) is \textbf{order complete}: For any two nonempty sets \(A, B \subset \mathbb{R}\) such that
    \[
        a \le b \quad \text{for all } a \in A,\; b \in B,
    \]
    there exists a number \(c \in \mathbb{R}\) satisfying
    \[
        a \le c \le b \quad \forall a \in A,\; b \in B.
    \]
}
\tsCor{1.1.6}{
    \textbf{The consequences of above axioms are:}
    \begin{enumerate}[noitemsep]
        \item Uniqueness of the additive and multiplicative inverse.
        \item $0 \cdot x = 0 \quad \forall x \in \mathbb{R}$.
        \item $(-1)\cdot x = -x \quad \forall x \in \mathbb{R}$, in particular $(-1)^2 = 1$.
        \item $y \ge 0 \iff (-y) \le 0$.
        \item $y^2 \ge 0 \quad \forall y \in \mathbb{R}$, in particular $1 = 1\cdot 1 \ge 0$.
        \item If $x \le y$ and $u \le v$, then $x + u \le y + v$.
        \item If $0 \le x \le y$ and $0 \le u$, then $xu \le yu$.
    \end{enumerate}
}
\tsThe{(Completeness of the real numbers)}{
    $\mathbb{R}$ is a commutative ordered field that is order-complete.
}
\tsCor{1.1.7 (Archimedean Principle)}{
    1. For \(x \in \mathbb{R}\) and \(y > 0\) there exists \(n \in \mathbb{N}\) such that
    \[
        ny > x.
    \]

    2. For every \(\varepsilon > 0\) there exists \(n \in \mathbb{N}\) such that
    \[
        \frac{1}{n} < \varepsilon.
    \]
}
\tsThe{1.1.8}{
    For every $t \ge 0$, $t \in \mathbb{R}$, the equation
    \(x^2 = t\)
    has a solution in $\mathbb{R}$.
}
\tsThe{(Supremum Infimum)}{
    1. Maximum and minimum (if exists) are unique \\
    2. Every non-empty subset bounded below (above) has a unique infimum (supremum). \\
    3. $S=\sup(X) \Leftrightarrow$
    $$(\large \forall x\in X: x\le S) \land (\large \forall\varepsilon>0\ \exists x\in X: \quad x>S-\varepsilon)$$
    4. $I=\inf(X) \Leftrightarrow$
    $$\large (\forall x\in X: x\ge I) \land (\forall\varepsilon>0\ \exists x\in X: \quad x<I+\varepsilon)$$
}
\\
\tsDef{(Maximum)}{
    Maximum is:
    \[
        \max\{x,y\}:=
        \begin{cases}
            x, & \text{if } y \le x, \\
            y, & \text{if } x \le y.
        \end{cases}
    \]
}
\tsDef{(Minimum)}{
    Minimum is:
    \[
        \min\{x,y\}:=
        \begin{cases}
            y, & \text{if } y \le x, \\
            x, & \text{if } x \le y.
        \end{cases}
    \]
}
\tsThe{(Properties of the absolute value)}{
    Properties:
    \begin{enumerate}
        \item $|x| \ge 0$.

        \item $|x+y| \le |x| + |y| \qquad$ (triangle inequality).

        \item $|x+y| \ge \big||x|-|y|\big|$ (inv. triangle inequality).
    \end{enumerate}
}
\tsThe{(Young's inequality)}{
    For every $\varepsilon > 0$ and all $x,y \in \mathbb{R}$ it holds:
    \[
        2|xy| \le \varepsilon x^2 + \frac{1}{\varepsilon} y^2.
    \]
}