\subsection{Complex Numbers}
\tsDef{(Complex number)}{
    A complex number is

    \[
        z=a+ib,\quad a,b\in\mathbb{R}
    \]
}
\\
\tsDef{(set of complex numbers)}{
    The set of complex numbers is

    \[
        \mathbb{C}=\{a+ib\mid a,b\in\mathbb{R}\}
    \]
}
\\
\tsDef{(Terminology of complex numbers)}{
    Terminology
    \begin{itemize} [noitemsep]
        \item $a=\operatorname{Re}(z)$ --- real part
        \item $b=\operatorname{Im}(z)$ --- imaginary part
        \item $i^2=-1$
        \item If $a=0$, the number is purely imaginary
    \end{itemize}
}
\tsDef{(Complex numbers Addition and Subtraction)}{
    Let $z_1=x_1+iy_1$, $z_2=x_2+iy_2$.
    \[
        z_1+z_2=(x_1+x_2)+i(y_1+y_2)
    \]

    \[
        z_1-z_2=(x_1-x_2)+i(y_1-y_2)
    \]
}
\\
\tsDef{(Complex numbers Multiplication)}{
    Let $z_1=x_1+iy_1$, $z_2=x_2+iy_2$.
    \[
        z_1z_2=x_1x_2-y_1y_2+i(x_1y_2+y_1x_2)
    \]
}
\\
\tsDef{(Complex conjugate)}{
    For a complex number $z = x + iy$, the complex conjugate of $z$ is given by
    \[
        \overline{z} = \overline{x + iy} = x - iy.
    \]
}
\\
\tsDef{(Complex numbers Division)}{
    To divide numbers we use conjugate:
    \[
        \frac{z}{w}=\frac{z\overline{w}}{w\overline{w}}=\frac{z\overline{w}}{|w|^2}
    \]
}
\\
\tsDef{(Properties of the Conjugate)}{
    Properties are:
    \begin{itemize} [noitemsep]
        \item $ z\overline{z}=|z|^2$
        \item $ z+\overline{z}=2\operatorname{Re}(z),\quad z-\overline{z}=2i\operatorname{Im}(z)$
        \item $ \overline{\overline{z}}=z$
        \item $\overline{z\pm w}=\overline{z}\pm\overline{w}$
        \item $\overline{zw}=\overline{z}\,\overline{w}$
        \item \(z = \overline{z}\), if \(z \in \mathbb{R}\)
        \item \(\overline{z} = -z\), \(z\) is purely imaginary
    \end{itemize}
}
\tsDef{(Modulus)}{
    The magnitude $|z|$ of a complex number $z = a + ib$ is the length of the corresponding vector.
    \[
        |z|=\sqrt{a^2+b^2}
    \]
    The distance between two complex numbers z1 and z2 is
    \[
        d(z_1,z_2)=|z_2-z_1|
    \]
    Triangle inequality works also for complex numbers:
    \[
        |z+w|\le |z|+|w|
    \]
}
\\
\tsDef{(Euler's Formula)}{
    The exponential function can be represented by a series expansion
    as
    \[
        e^x=\sum_{k=0}^{\infty}\frac{x^k}{k!}
    \]
    If we substitute $x = it$ into this formula, we obtain Euler's formula.
    \[
        e^{it}=\cos t+i\sin t
    \]
    We can write $sin(t)$ and $cos(t)$ using complex
    exponential functions.
    \[
        \cos (t)=\frac{e^{it}+e^{-it}}{2}
    \]

    \[
        \sin (t)=\frac{e^{it}-e^{-it}}{2i}
    \]
}
\\
\tsDef{(Exponential with Complex Argument)}{
    We can also allow complex arguments for the exponential function. Then the following rules apply:
    \begin{itemize} [noitemsep]
        \item \(e^{z+w}=e^ze^w\)
        \item \(e^{a+ib}=e^a(\cos b+i\sin b)\)
        \item \(e^{z+i2\pi}=e^z\)
    \end{itemize}
}
\tsDef{(Polar Form)}{
    A complex number can also be described by specifying the distance from the origin and by specifying a suitable angle (\textbf{polar form}):
    \(
    z=re^{i\varphi}.
    \)

    \begin{itemize}[noitemsep]
        \item $r=|z|\ge0$ - distance
        \item $\varphi\in(-\pi,\pi]$ - polar angle
        \item $\varphi=\arg(z)=\text{arc}(z)$
    \end{itemize}
}
\tsDef{(Polar to Cartesian conversion)}{
    Polar to Cartesian conversion is:
    \begin{itemize} [noitemsep]
        \item $x=r\cos\varphi$
        \item \(y=r\sin\varphi\)
    \end{itemize}
}
\tsDef{(Cartesian to Polar conversion)}{
    Cartesian to Polar conversion is:
    \begin{itemize} [noitemsep]
        \item \(r=\sqrt{x^2+y^2}\)
        \item \(\varphi=\arctan\frac{y}{x}\)
    \end{itemize}
    If $x=0$:
    \[
        \arg(iy)=
        \begin{cases}
            \frac{\pi}{2},  & y>0 \\
            -\frac{\pi}{2}, & y<0
        \end{cases}
    \]
}
\tsDef{(Powers of Complex Numbers)}{
    The following rules must be observed:

    \begin{itemize}[noitemsep]
        \item \(z_1z_2=r_1r_2e^{i(\varphi_1+\varphi_2)}\)
        \item \(\frac{z_1}{z_2}=\frac{r_1}{r_2}e^{i(\varphi_1-\varphi_2)}\)
        \item \(z^n=r^ne^{in\varphi}\)
    \end{itemize}
}
\tsDef{(Roots from complex numbers)}{
    For $n \in \mathbb{N}$, the solutions to the equation \(z^n = re^{i\varphi}\) are
    \[z_k=r^{1/n}e^{i(\varphi/n+2\pi k/n)},\quad k=0,\dots,n-1\]
}
\\
\tsDef{(Quadratic Equations)}{
    For the quadratic equation
    \[
        az^2+bz+c=0,
    \]
    in complex numbers the solution is still
    \[
        z_{1,2}=\frac{-b\pm\sqrt{b^2-4ac}}{2a}.
    \]
}
\\
\tsDef{(Fundamental Theorem of Algebra)}{
    Each polynom
    \[
        p(z)=\sum_{k=0}^n a_k z^k,\quad a_k\in\mathbb{C}
    \]
    can be factored into linear factors, i.e. written as
    \[
        p(z)=a_n(z-z_1)(z-z_2)\dots(z-z_n)
    \]
    The numbers $z_k$ are therefore precisely the zero points of $p(z)$ (with \textbf{multiplicity}).
}
\\
\tsCor{(non-real roots apper in conjugate pairs)}{
    Non-real roots of a polynomial with \textbf{real} coefficients occur in conjugate pairs
    \(
    z,\overline{z}.
    \)
}